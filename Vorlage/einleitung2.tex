\section{Einleitung} \raggedbottom 
Der Major Histocompatibility Complex, kurz MHC, ist ein Teilstück der DNA von Wirbeltieren, welches unter anderem eine tragende Rolle bei Vorgängen des Immunsystems besitzt. Durch seine große Variabilität dient es als Ausweis der körpereigenen Zellen, um sich vor dem Immunsystem von Fremdgewebe zu unterscheiden. Unter anderem ergibt sich daraus ein großes Interesse für Organtransplantationen, die konkrete Sequenz der MHC-Region zu kennen. Ferner ist der MHC der Teil des Genoms, mit der größten Relevanz für Erbkrankheiten. 
Leider resultiert aus der großen Variabilität ein eben so großes Problem bei der Analyse dieses Aufbaus. Um dieses Problem zu verstehen, müssen wir erst verstehen, wie bei der Assemblierung, also der Bestimmung der DNA-Sequenz, vorgegangen wird.\\

Etablierte DNA-Sequenzierungstechnologien ermöglichen keine direkte Ermittlung der Basenpaare bei langen DNA-Sequenzen. Daher werden sie in kürzere Stücke zerlegt, für die dann die Basenpaare bestimmt werden können. Diese kürzere Stücke heißen \emph{Reads}. Für das Erstellen solcher Reads gibt es verschiedene Verfahren, die alle verschiedene Vor- und Nachteile haben. So werden bei der sogenannten Illumina-Sequenzierung sehr kleine Reads mit einer Länge von ungefähr 200 Basenpaaren erzeugt. Diese können sich teilweise überlappen, wodurch es möglich ist, einige Reads zu so genannten \emph{Contigs} zusammenzufügen. Im Idealfall passen die Reads so gut zueinander, dass der gesamte untersuchte DNA-Bereich rekonstruiert werden kann. Ein Fall, bei dem dies nicht möglich ist, zeigt diese Situation:
\begin{align*}
\text{ATTAAGCCTTAGGGTTATATCATATA}&\text{TATATATA}\\
&\text{TATATATATATGTAAGCGTTCGTTGTCTC}
\end{align*}
\begin{figure}[b]
\begin{center}
\includegraphics[width=0.9\textwidth]{bilder/repeat}
\end{center}
\caption{Repeat in einer Sequenz}
\label{repeat}
\end{figure}
Durch den sehr simplen Aufbau des Zwischenbereiches aus abwechselnden Adenin- und Thyminbasen ist es nicht möglich, die genaue Positionierung zu bestimmen. Auch wenn die Contigs nicht exakt zueinander positioniert werden können, so ist dies zumindest ungefähr möglich. Noch größere Schwierigkeiten machen Regionen in der DNA, die über eine längere Strecke komplett identisch sind, sogenannte Repeats. Das Problem wird in Abbilding \ref{repeat} verdeutlicht. Zu sehen sind fünf Contigs, wobei ein Pfeil von Contig $a$ zu Contig $b$ verdeutlicht, dass in der DNA Contig $a$ direkt vor Contig $b$ kommt. Da Contig 63 (blau) zwei mal in der DNA vorkommt, ist es nicht trivial zu bestimmen, wie der Strang verläuft. Und MHC ist sehr repetitiv, besitzt also viele Wiederholungen. Daher ist diese Methode ungeeignet um MHC zu assemblieren.


Eine weitere Möglichkeit der Sequenzierung ist die Nanopore-Sequenzierung. Hierbei können sehr lange Reads von über 10\,000 Basenpaaren gelesen werden. Aufgrund des Längenunterschieds werden die Reads aus der Illuminar Sequenzierung auch \emph{Short-Reads} genannt, und die Reads aus der Nanopore-Sequenzierung \emph{Long-Reads}. Durch ihre Länge umfassen die Long-Reads oftmals bereits vollständige Repeats plus Umgebung. In unserer Abbildung entspräche dies einem Read, in dem Contig 132, Contig 63 und Contig 32 (gelb, blau und grün) Teil von einem Read sind und es keine Schwierigkeiten diesbezüglich gibt. Leider ist die Fehlerrate bei dieser Methode sehr hoch. 
So werden sehr viele Long-Reads aus der selben Region benötigt, um die richtigen Basenpaare verlässlich zu bestimmen.


Daher hat sich die Manchot-Forschungsgruppe vom Institut für Medizinische Mikrobiologie und Krankenhaushygiene der HHU, mit deren Zusammenarbeit diese Arbeit entstand, eine Kombination der beiden Methoden entwickelt.
% Diese deren Nachteile ausgleichen soll. 
Die verlässlichen Contigs aus der Illuminar-Sequenzierung wurden auf die Long-Reads gemappt und so deren Abstände zueinander bestimmt. Betrachten wir zur Anschauung Abbildung \ref{longread} eines Long-Reads. Die kleinen rot eingefärbten Bereiche stellen Fehler dar, die bunten Bereiche stellen Gebiete dar, in denen die Basenpaarsequenzen mit denen eines Contigs aus den Illuminar-Daten übereinstimmen. In der selben Situation wie in Abbildung \ref{repeat} hätten wir nun die Information erhalten, dass Contig 114 und Contig 32 zusammengehören, also die rechte Auflösung der Abbildung richtig ist. 
\begin{figure}
\begin{center}
\includegraphics[width=0.8\textwidth]{bilder/longread}
\end{center}
\caption{Ein Long-Read über mehrere Contigs}
\label{longread}
\end{figure}
Die Manchot-Forschungsgruppe hat diese Daten gesammelt und daraus eine Liste aus paarweisen Daten extrahiert. Diese besitzt die Form: Contig $a$ hat zu Contig $b$ die Entfernung $d$ (in Anzahl von Basenpaaren zwischen $a$ und $b$). Dieses Dreiertuple aus zwei Contigs und einer Distanz nennen wir einen \emph{Constraint}.


Es bleiben noch einige Schwierigkeiten für die Assemblierung zu beachten. Die Distanzwerte zwischen den Contigs sind meistens durch Fehler in den Long-Reads verfälscht. Dadurch sind erst mehrere Constraints, die eine ähnliche Distanz zwischen zwei Contigs prognostizieren, wirklich belastend. Bis zu welchen Distanz sich Constraints noch bestätigen und ab wann sie sich widersprechen ist hierbei ein entscheidender Aspekt, der betrachtet werden muss. Die Repeats lassen sich nicht immer so eindeutig auflösen wie in unserem Beispiel. In manchen Fällen kann erst im Gesamtzusammenhang erkannt werden, wie der Strang verläuft. Letztlich bleibt noch die große Datenfülle als Herausforderung zu nennen: Bei rund 122\,000 auftretenden Distanzen zwischen 2\,124 Contigs ist eine manuelle Zusammenfügung nicht zielführend und mindestens eine Teilautomatisierung der Prozesse obligatorisch. Auf der anderen Seite sind die Constraints nicht gleichmäßg verteilt, sodass es Regionen gibt, bei denen mit sehr wenig Informationen ausgekommen werden muss.

Hier stellt sich die Frage, mit welchen Methoden diese Probleme bewältigt werden können. Denkbar wäre eine Umsetzung mittels Linearer Programmierung. Das Ziel dieser Arbeit wird sein, zu untersuchen, ob lineare Programmierung hierbei anwendbar ist, und welche Vor- und Nachteile lineare Programme mit sich bringen.
