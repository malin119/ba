% Definieren Sie hier Ihre Makros



% Setzen Sie hier die folgenden Makros: Name, Titel und Typ der Arbeit
\newcommand{\Autor}{Marvin Lindemann}
\newcommand{\Titel}{Contig-Assembly der MHC-Region mittels
Linearer Programmierung}
\newcommand{\Typ}{Bachelorarbeit}





% Einige Beispiele, die vielleicht interessant sind:
% deutsche Bezeichnung
\floatname{algorithm}{Algorithm}
\newtheorem{prop}{Proposition}[chapter] 
\newtheorem{lem}[prop]{Lemma}
\newtheorem{thm}[prop]{Theorem}
\newtheorem{kor}[prop]{Corollary}
\newtheorem{beobachtung}[prop]{Beobachtung}
\theoremstyle{definition} 
\newtheorem{defi}[prop]{Definition}
%\newtheorem*{defi}{Definition} 
\newtheorem{beh}[prop]{Claim}
\newtheorem*{beh*}{Claim}
\theoremstyle{remark} 
\newtheorem*{rem}{Remark}
\newtheorem{bsp}[prop]{Example}
\newenvironment{bew}{\begin{proof}[Proof. ]}{\end{proof}}
\newenvironment{ohne}{\begin{proof}[Ohne Beweis.]}{\end{proof}}

% Ausgleichsspalten fuer tabularx
\newcolumntype{L}{>{\raggedright\arraybackslash}X}
\newcolumntype{R}{>{\raggedleft\arraybackslash}X}
\newcolumntype{C}{>{\centering\arraybackslash}X}

% Komplex-Klasse
\newcommand{\Klasse}[1]{\text{\mbox{\rm\bf #1}}}

\newcommand{\NachbarSym}{\mathcal{N}}
\newcommand{\Nachbar}[1]{\mathcal{N}(#1)}

% Problem
%\newcommand{\Problem}[1]{\text{\mbox{\rm\it #1}}}
\newcommand{\Problem}[1]{\text{\mbox{\rm\scshape #1}}}
\newcommand{\DecProblem}{\ensuremath{\mathcal{D}}}
\newcommand{\Len}{\ensuremath{\texttt{len}}}
\newcommand{\Max}{\ensuremath{\texttt{max}}}

% Kurzbezeichnungen
\newcommand{\gdw}{\ensuremath{\ \Leftrightarrow \ }}
\newcommand{\GDW}{\ensuremath{\ \Longleftrightarrow \ }}
\newcommand{\entspr}{\ensuremath{ \widehat{=} }}
\newcommand{\inv}[1]{\ensuremath{ {#1}^{-1} }}
\newcommand{\manyone}{\ensuremath{\mbox{$\,\leq_{\rm m}^{{\Klasse{p}}}$\,}}}
\newcommand{\p}{\Klasse{P}}
\newcommand{\np}{\Klasse{NP}}

% Definitition von einem Problem, 5 Parameter
% 1: Name des Problems
% 2: Label zum Referenzieren
% 3: Gegeben:
% 4: Gesucht:
% 5: Abk. in Klammern hinter dem Namen (optional)
\newcommand{\Problemdef}[5]{
    \begin{center}\label{#2}\small\begin{tabularx}{0.9\textwidth}{lL}\toprule
            \multicolumn{2}{c}{{\Problem{#1} }
                \ifthenelse{\equal{#5}{}}{}{~(\Problem{#5})}}\\ \hline
            \textbf{Given:}&{#3}\\
            \textbf{Question:}&{#4}\\ \bottomrule
\end{tabularx}\end{center}}


\newcommand{\Optiproblemdef}[5]{
    \begin{center}\label{#2}\small\begin{tabularx}{0.9\textwidth}{lL}\toprule
            \multicolumn{2}{c}{{\Problem{#1} }
                \ifthenelse{\equal{#5}{}}{}{~(\Problem{#5})}}\\ \hline
            \textbf{Given:}&{#3}\\
            \textbf{Output:}&{#4}\\ \bottomrule
\end{tabularx}\end{center}}

%\newcommand{\Problemdef}[5]{
%\begin{center}\label{#2}\begin{tabularx}{0.75\textwidth}{lL}\hline\hline
%\multicolumn{2}{c}{{\Problem{#1} }
%\ifthenelse{\equal{#5}{}}{}{~(\Problem{#5})}}\\ \hline
%\textbf{Gegeben:}&{#3}\\
%\textbf{Frage:}&{#4}\\ \hline\hline
%\end{tabularx}\end{center}}


% Weitere Makros:
\newcommand{\bb}{\ensuremath{\mathbb{B}}}
\newcommand{\nn}{\ensuremath{\mathbb{N}}}
\newcommand{\zz}{\ensuremath{\mathbb{Z}}}
\newcommand{\qq}{\ensuremath{\mathbb{Q}}}
\newcommand{\rr}{\ensuremath{\mathbb{R}}}

\newcommand{\npc}{$\np$\text{-complete}}

\newcommand{\conp}{\Klasse{coNP}}
\newcommand{\conpc}{\Klasse{coNP}\text{-complete}}
\newcommand{\conph}{\Klasse{coNP}\text{-hard}}

%%%%%%%%%%%%%%%%%%%%%%%%%%%%%%%%%%%%
% Algorithm-Umgebung
%%%%%%%%%%%%%%%%%%%%%%%%%%%%%%%%%%%%
\renewcommand{\algorithmicrequire}{\textbf{Eingabe:}}
\renewcommand{\algorithmicensure}{\textbf{Ausgabe:}}